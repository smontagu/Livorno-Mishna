\newcounter{perekCount}
\newcounter{halachaCount}[perekCount]
\newcounter{tractateCount}
\newcounter{perakim}

% \tractatehead{title for printing}{hyperlink reference}
\newcommand{\tractatehead}[2]{%
 \cleardoublepage
 \setcounter{perekCount}{0}
 \thispagestyle{empty}
 \def \tractateTitle{#1}
{\centering\LARGE\bfseries \hypertarget{#2}{#1}

  }}

% \summarycontentsline{tractate name}{hyperlink reference}
%                     {number of chapters [numeric]}
\newcommand{\summarycontentsline}[3]{%
\setcounter{perakim}{#3}%
\stepcounter{tractateCount}%
\textlarger{\alph{tractateCount}} \hyperlink{#2}{#1} \alph{perakim} פרקים\hdot%
}

\newcommand{\summarycontents}[1]{
  \centerlastline
  \foreach \linkText / \linkName / \numChapters in #1 {
      \summarycontentsline{\linkText}{\linkName}{\numChapters}
  } 
}

% \sederContents{title of seder}{number of tractates [hebrew]}
%               {list of tractates}{total number of chapters [hebrew]}
\newcommand{\sederContents}[4]{
  {
    \centering  
    {\Large משניות סדר #1 }
 
  יש בו #2 מסכתות וזהו סדורן

    \begin{minipage}{0.9\textwidth}
    \small
    \summarycontents{#3}
    בין הכל \textlarger[2]{#4} פרקים׃
    \end{minipage}

  }
}

% centered rubric
\newcommand{\crubric}[1]{{\centering\small\bfseries{#1}

  }}

% centered header (keep with next paragraph)
\newcommand{\chead}[1]{%
  \vspace{0.25em}
  \needspace{3\baselineskip}
  \crubric{#1}
}

% centered footer
\newcommand{\cfoot}[1]{%
  \vspace{0.25em}
  \needspace{3\baselineskip}
  \crubric{#1}
}

\newcommand{\mispar}[1]{{\footnotesize\textsuperscript{\alph{#1}}}}

% There is no Unicode character for the raised dot used in many Hebrew books
% as mid-sentence punctuation. This is a hack to replace it using
% U+005C4 HEBREW MARK UPPER DOT
\newcommand{\hdot}{\raisebox{-0.45\height}{^^^^00a0^^^^05c4}\spacefactor=501{} }

% \perek[tolerance]{first word}{rest of first halacha}
% First halacha in a perek
\newcommand{\perek}[3][0]{%
 \stepcounter{perekCount}
 \perekbar[#1]{#2}{#3}{פרק \alph{perekCount}}
}
  
% \baraita[tolerance]{first word}{rest of first halacha}
% First halacha in a baraita (e.g. after the last perek of Bikkurim)
\newcommand{\baraita}[3][0]{%
 \setcounter{halachaCount}{0}
 \perekbar[#1]{#2}{#3}{ברייתא}
}

% \perekbar[tolerance]{first word}{second word}{rest of first halacha}{heading}
% Called by \perek and \baraita
\newcommand{\perekbar}[4][0]{%
 \vspace{0.5em}
 \stepcounter{halachaCount}
 \needspace{2\baselineskip}
 \crubric{#4}

 \tolerance #1
 \centerlastline
 \leadword{#2}#3
}

% \halacha[tolerance]{text}
% Prints a halacha
\newcommand{\halacha}[2][0]{%
 \tolerance #1
 \stepcounter{halachaCount}
 \mispar{halachaCount}
 \centerlastline
 #2\nowidow
}

\newcommand{\selik}{~~{\footnotesize סליק}}

\newenvironment{tp}[1]{% tolerant paragraph}
  \par\tolerance=#1\relax
}{%
  \par
}

