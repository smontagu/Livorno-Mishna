\setcounter{page}{1}
\tractatehead{הקדמה}{introduction}

{
  \centering
  \large
  שנמצאת במשניות שהיה לומד מעלת מורינו הרב המדקדק רבי דוד בכ״ר שלמה אלטארטס זצ״ל שעשאה לתיקון המשניות בחייו וזה לשונה
}

{\centering\LARGE\bfseries עם שונה שלום

}

{
  \small
  \centerlastline
  \leadword{ידיע}{להוי}לך שונה ידיד דתהלות לאל יתברך כעל אשר גמלני ורב טוב שהחייני
  וקיימני לגמור בעדי מה שבצפייתי צפיתי ואויתי בעודי בחיים חייתי להיות
  לי למגן וצנה משכיות סדורות ערוכות בכל ושמורות מוגהות ביתר שאת כיד ה׳
  הטובה אנה לידי\hdot
  והוא יודע ועד כמה יגיעות יגעתי עד שבאתי לכלל ישובן
למוצאן ולמובאן והביאו לי מקום הראשונים שבקדושה להתגדר זעיר שם זעיר שם
ו{\larger דוד}
    מכרכר בכל עז לפני ה׳ מטפס ועולה מטפס ויורד לו לשמו כי כל ישעי וכל
חפצי לדבקה בו׃

\tolerance 1783
\leadword{ואלה}{פקודיהם}למסעיהם בראשונה השתדלתי למצוא כל הגירסאות הטובות
והבריאות ש״ס ורבינו שמשון זצ״ל על המשנה ומבעל הערוך ז״ל זולת מה
שהביאו המפרשים הקדושים אשר בארץ המה הרמב״ם ז״ל והר״ב ז״ל ובעל תי״ט
כי מימיהם אנו שותים׃

\tolerance 116
\leadword{ועוד}{שנית}בפסק ההלכות מה שלא פסקו הם ז״ל לא שלותי עד ששקטתי לברר
ההלכה על תלה וארמונה על משפטו ישב\hdot
כי דפקתי על דלתי רבינו הגדול
פום ממלל רברבן הרמב״ם ז״ל בחבורו הגדול ומהולל בתשבחות שם העליתי מצודתי
ומצאתי און לי והוצאתי לאור משפטי\hdot
ומפוסקים אחרים\hdot
ולאן מלתא זוטרתי׃

\tolerance 1024
\leadword{הן}{אמת}חפצתי בטו״חות חכמה כי מצאתי קצת סתירות להרב ז״ל מיניה וביה
מהפי׳ והחיבור לא רציתי לשלוח יד להכריע\hdot
הגם כי כבר ראיתי לאחרונים
ז״ל כבת קול יוצאת מפיהם ומפי כתבם דהלכתא כמילי בתראי דפסק בחבורו
דלסוף ימיו עשהו ויכוננהו ואזן וחקר ותקן ותכן במאזני שכלו הזך כי מי כמוהו
מורה אב הוא דלא לוסיף עליהם עם כל זה האיש הנלבב יעשה מה שלבו חפץ הוא
יבחר ולא אני מכריע שאין בו ממש\hdot
א״כ איפה ראה כל היכא דתמצא דמות כוכב
כזה * בא האות והמופת דהלכתא כוותיה\hdot
ואי איכא פלוגתא דרבוותא על הפסק
דין מר פסיק כוותיה ומר לא פסיק כוותיה או הרמב״ם ז״ל דהדר ביה בחבורו
עשיתי ציון כזה
{\raisebox{-.5\height}{ ^^^^05af ^^^^0592}}
האות הראשון הוא מהפירוש והשני מהחבור ודוק׃

\leadword{ותו}{משכחת}לה דלפעמים במשנה איכא פלוגתא דתנאי ולית הלכתא כשום חד
מנייהו משום דאיכא פסק הלכה כי סברא שלישית דאוקי ש״ס דבתר דידהו
סמיכנא\hdot
ואמשול לך משל בגיטין פ״ז מ״ד לא תתיחד עמו אלא בפני עדים וכו׳
מה היא באותן הימים ר׳ יהודה אומר כאשת איש לכל דבריה ר׳ יוסי אומר מגורשת
ואינה מגורשת\hdot
לית הילכתא לא כרבי יהודה ולא כר׳ יוסי לפי שיש פסק דין בזה
דמגורשת לגמרי והר״ב ובעל תי״ט השמיטוהו ויעזבוהו ולו נטשתוהו\hdot
ובחפוש אחר
חפוש באמתחות פי׳ הר״מ מצאתיו כי הוא כרבא דכולי ביה\hdot
וזולתם הרבה\hdot
והכל
רשום בכתב אמת כי שניתי ושלשתי לעשות ציונים מכוונים במלאכת השמים׃

\leadword{ועוד}{בה}שלישיה כי בדקתי עד שידי מגעת בש״ס וכל איכא דמשכחיה דאוקי ש״ס
ההוא מתניתין בחסורי מחסרא עשיתי רושם על פני חוץ לכתוב שתי תיבות
כזא חסורי מחסרא להיותם לי לזכרון בין עיני וטוטפות ולסנבוטין כי זה חלקי
מכל עמלי\hdot
ומצאתי להר״ב ז״ל דלפעמים הכי קאמר שבש״ס קורא אותו ח״מ וליתא
בנשיקת ידיו ורגליו דח״מ תוספת על המשנה בפועל והכי קאמר בכח כידוע המשל
בפרק המניח במשנת כ׳ שהיו מהלכים בר״הר אחד רץ ואחד מהלך או שהיו שניהם
רצים כתב הר״ב ח״מ והכי קאמר וכו׳ ועיין מ״ש דף ל״ב דקאמר התם השתא אחד
רץ ואחד מהלך פטור שניהם רצים מבעיא אלא הכי קאמר וכו׳ עמוד עליו כי האמת
יורה דרכו\hdot
ושאר פרפראות והבנות בקיצור על קצת משניות ודיוקים בדקדוק
הלשון על הקריאה כאשר עיני השונה תחזינה מישרים ומה שימצא בה דבר טוב יתלה
הדבר במי שהוא טוב וישר ה׳ כי הוא הנותן לפתאים ערמה וכח לעשות חיל ומה
שלא יצדק כי לא בא אל נכון יתלה הדבר בי לפי קוצר המשיג כי נער אנכי נער
ישראל ואוהבהו עושה מעשה נערות\hdot
וה׳ יכפר בעדי אם שגיתי ואתי תלין משוגתי
כי אני מכיר את מקומי שאיני כדאי לשום דבר יציב ונכון שאין בי לא חכמה ולא
תבונה עת לחננה\hdot
קריאת חנה דוד הוא הקטן ולא יבצר ממך מזימה שונה מתנות
שנחלן אל לעמו ישראל כי תקנתי מעוות יוכל לתקון בחליף וחסיר ויתיר שגיאות
הבאות אל שער הדפוס למען לא אטה ימין ושמאל קולע אל השערה ולא אחטיא
בגרסת המשנה בעזר המאזרני חיל לתת תמים דרכי ואעידה לי שנים עדים נאמנים
כי צרפתים כצרוף את הכסף וכבחון את הזהב הא׳ בפרק ד׳ דביצה ואין חותכין את
הפתילה לשנים רי״א חותכה באור כצ״ל ולא יותר ובכל הגרסאות חדשות גם ישנות
נמצא דתני לב׳ נרות וטעות הוא בידם דבש״ס הכי קאמר ואין חותכין את הפתילה
לשנים\hdot
מ״ש בסכין דלא דקמתקן מנא באור נמי קא מתקן מנא תני ר׳ חייא חותכה
באור בפי שתי נרות ע״כ\hdot
ופירש רש״י דקמתקן מנא א׳ היתה ועושה אותה שנים\hdot
בפי שתי נרות שתי ראשיה בתוך ב׳ נרות אם צריך להדליקן כאחד ומדליק באמצע
דלא מוכח דלתקוני מנא מכוין אלא להדלקה בעלמא עכ״ל הא קמן דלא גרסינן
במתניתין לשתי נרות דבתוספתא תני הכי ר׳ חייא ולזה כוין ר׳ יהודה בכח כשאמר
חותכה באור וק״ל\hdot
ובעל תי״ט לא קאמר ביה מידי\hdot
והב׳ בפרק ב׳ דחולין השוחט
לשם זבחים כצ״ל ובגרסאות שלנו תני לשם שלמים והיא שגגה שיוצאת שפירוש
זבחים כאן ר״ל שלמים ואינו שם כולל כמו בריש זבחים כל הזבחים שנזבחו והביאו
בעל תי״ט ז״ל ומצאתיו מתוקן בגירסת הש״ס דפוס אמשטרדם\hdot
אם כן ל״ג שלמים
דהוו סברי בעלי הדפוס דשלמים עדיפא להו דהוא מפורש טפי\hdot
וכי תורת המשנה
זאת תורת העולה על רוחה כדסברא דעתייהו וכהנה רבות עמי אשר שם אלהים
בפי לדבר\hdot
כי הכל עשיתי לזכות את עצמי שלא לגשש כעורים קיר שאיני כדאי
וראוי לזכות את אחרים כי מה אני להבל דמה תולעת ולא איש לדייק בלישניה לישנא
קלילא הראוי לבילה פרט הנושר בשעת הבצירה דערבא ערבא צריך והבא ללמד
ונמצא למד אבל בטחתי בחסד אלהים יעזרני על דבר אמת לאמתו (כי פה לא
עשיתי מאומה והבור רק אין בו מים) ויוציאני למרחב יחלצני לתת לי לחם ושמלה
שאוכל להוציא לאור מה שיש בלבי לשמור ולעשות כי כל יוכל\hdot
ולא יזכור לי עון
אשר חטא כי הוא ירבה לסלוח\hdot
ותקרב ותבואה עת לחננה כי בא מועד תרבה
הדעת כמים לים מכסים בקול ששון ובקול שמחה אכי״ר :

}

\vspace{1em}

\setlength{\epigraphwidth}{0.6\textwidth}
\setlength{\epigraphrule}{0pt}
\epigraphposition{flushleft}
\epigraphtextposition{center}
\epigraphsourceposition{center}

\epigraph{לישנא קלילא והראוי לבילה}{{\larger[2] דב״ש אלטאראס} זלה״ה}
