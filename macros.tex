\newcounter{perekCount}
\newcounter{halachaCount}[perekCount]
\newcounter{tractateCount}
\newcounter{perakim}

% \tractatehead{title for printing}{hyperlink reference}
\newcommand{\tractatehead}[2]{%
 \cleardoublepage
 \setcounter{perekCount}{0}
 \thispagestyle{empty}
 \def \tractateTitle{#1}
{\centering\LARGE\bfseries \hypertarget{#2}{#1}

  }}

% \summarycontentsline{tractate name}{hyperlink reference}
%                     {number of chapters [numeric]}
\newcommand{\summarycontentsline}[3]{%
\setcounter{perakim}{#3}%
\stepcounter{tractateCount}%
\textlarger{\alph{tractateCount}} \hyperlink{#2}{#1} \alph{perakim} פרקים\hdot%
}

\newcommand{\summarycontents}[1]{
  \centerlastline
  \foreach \linkText / \linkName / \numChapters in #1 {
      \summarycontentsline{\linkText}{\linkName}{\numChapters}
  }
}

% \sederContents{number of tractates [hebrew]}
%               {list of tractates}{total number of chapters [hebrew]}
\newcommand{\sederContents}[3]{
  {
    \centering
    {\Large משניות \sederTitle}

  יש בו #1 מסכתות וזהו סדורן

    \begin{minipage}{0.9\textwidth}
    \small
    \summarycontents{#2}

    בין הכל \textlarger[2]{#3} פרקים׃
    \end{minipage}

  }
}

% centered rubric
\newcommand{\crubric}[1]{{\centering\small\bfseries{#1}

  }}

% centered header (keep with next paragraph)
\newcommand{\chead}[1]{%
  \vspace{0.25em}
  \needspace{4\baselineskip}
  \crubric{#1}
}

% centered footer
\newcommand{\cfoot}[1]{%
  \vspace{0.25em}
  \crubric{#1}
}

\newcommand{\mispar}[1]{{\footnotesize\textsuperscript{\alph{#1}}}}

% There is no Unicode character for the raised dot used in many Hebrew books
% as mid-sentence punctuation. This is a hack to replace it using
% U+005C4 HEBREW MARK UPPER DOT
\sfcode`\^^^^05c43000
\newcommand{\hdot}{\raisebox{-0.5\height}{^^^^2009^^^^2009^^^^05c4} }

\tolerance=8192

% \perek{first word}{rest of first halacha}
% First halacha in a perek
\newcommand{\perek}[2]{%
 \stepcounter{perekCount}
 \perekbar{פרק \alph{perekCount}}{#1}{#2}
}

% \baraita[subhead]{first word}{rest of first halacha}
% First halacha in a baraita (e.g. after the last perek of Bikkurim)
\newcommand{\baraita}[3][]{%
 \setcounter{halachaCount}{0}
 \perekbar[#1]{ברייתא}{#2}{#3}
}

% \perekbar[subhead]{heading}{first word}{rest of first halacha}
% Called by \perek and \baraita
\newcommand{\perekbar}[4][]{%
 \vspace{0.5em}
 \stepcounter{halachaCount}
 \needspace{2\baselineskip}
 \crubric{#2}

 \centerlastline
 #1

 \leadword{#3}#4
}

% \halacha{text}
% Prints a halacha
\newcommand{\halacha}[1]{%
 \stepcounter{halachaCount}
 \mispar{halachaCount}
 \centerlastline
 #1\nowidow
}

\newcommand{\selik}{~~{\footnotesize סליק}}

% \titlepage{Hebrew chronogram for year of issue}
%           {Gregorian year of issue}
\newcommand{\titlepage}[2]{%
\Centering

  {\huge\bfseries משניות}

  {\LARGE\bfseries\sederTitle}

  \vspace{1em}

  \vspace{0.5em}
  על פי מהדורת הרב דוד אלטאראס ז״ל\\שהודפס בשנת ה׳תק״ע בעיר פיסא

    \vspace{8em}
    פה {\LARGE ירושלים} יע״א

    {\small שנת} #1 {\small לפ״ק}

    \vspace{2em}
      \includegraphics[width=10mm]{hatafSegolLogoNoText.png}\\

    \vspace{.5em}
  {
    \bfseries הוצאת חטף־סגול\\#2

 {\footnotesize  הובא לדפוס באמצעות \XeLaTeX\ ע״י שמעון מונטגיו}

  }

}

% code for special handling of specific numbers by "Circumscribe" copied from https://tex.stackexchange.com/questions/300008/modify-specific-hebrew-alpha-numerals-on-page-number
% with a few modifications to the function names
\makeatletter
\let\@hebrew@numeral@usual\@hebrew@numeral  %% <- store old definition
\newcommand*\@hebrew@numeral@special[1]{%   %% <- new definition
  \ifcsdef{specialnum@\number#1}            %% <- if this number is listed
    {\csuse{specialnum@\number#1}}          %% <- give it special handling
    {\@hebrew@numeral@usual{#1}}%           %% <- otherwise use usual handling
}
\let\@hebrew@numeral\@hebrew@numeral@special %% <- replace old definition
\makeatother

%% Declare numbers for special handling:
\newcommand*\newspecialnum[2]{\csdef{specialnum@#1}{#2}}
\newspecialnum{16}{יו}    % At least some Livorno printings don't use טז for 16
\newspecialnum{116}{קיו}
\newspecialnum{216}{ריו}
\newspecialnum{316}{שיו}
\newspecialnum{416}{תיו}
\newspecialnum{516}{תקיו}
\newspecialnum{616}{תריו}
\newspecialnum{716}{תשיו}
\newspecialnum{816}{תתיו}
\newspecialnum{916}{תתקיו}
\newspecialnum{18}{חי}    % Add some other "specials" while we're here
\newspecialnum{270}{ער}
\newspecialnum{272}{ערב}
\newspecialnum{275}{ערה}
